\chapter*{Resumen}
\thispagestyle{empty}

El aprendizaje automático consiste en algoritmos que mejoran automáticamente con el tiempo mediante el uso de datos. Estos datos consisten en las entradas y salidas deseadas de un problema específico. Los algoritmos de aprendizaje automático aprenden cómo adaptar una red neuronal a estos datos de manera que cuando se les proporcionen datos similares, darán un resultado que debería ser correcto la mayor parte del tiempo.

La computación aproximada es un paradigma de diseño con el objetivo de mejorar la eficiencia energética y el rendimiento de ciertas aplicaciones a cambio de que los resultados de los cálculos tendrán un error asociado. Por lo general, la computación aproximada se usa en aplicaciones donde estos errores no son un factor decisivo y pueden tolerarse.

Dado que el aprendizaje automático es naturalmente un paradigma informático propenso a errores, tiene sentido explorar cómo encaja en el mundo de la computación aproximada. Este trabajo se centra en unir tanto el aprendizaje automático como la computación aproximada. Explora cómo herramientas y aceleradores neuronales disponibles en el mercado se pueden utilizar en ambientes de sistemas embebidos y de bajo consumo de potencia, para aprovechar las ventajas del aprendizaje automático en aplicaciones de computación aproximada y, por otro lado, cómo se pueden acelerar las aplicaciones de aprendizaje automático utilizando técnicas de computación aproximada.

En la primera parte de este trabajo se proponen transformaciones algorítmicas para aprovechar las características de un acelerador neuronal para mejorar el rendimiento de aplicaciones tolerantes a errores. Los resultados muestran que dos de las cinco aplicaciones probadas pueden acelerarse manteniendo un error por debajo del 10 \%.

La segunda parte de este trabajo muestra cómo algunas técnicas de computación aproximada pueden acelerar aplicaciones de aprendizaje automático a cambio reducir la precisión de la red neuronal ligeramente. También muestra cómo el uso de un acelerador neuronal puede mejorar hasta 10 veces la eficiencia energética de las aplicaciones de aprendizaje automático en un entorno de bajo consumo de energía.

\bigskip

% en español e inglés

\textbf{Palabras clave:} computación aproximada, aprendizaje automático, acelerador neuronal

\cleardoublepage

%%% Local Variables: 
%%% mode: latex
%%% TeX-master: "main"
%%% End: 
