\chapter{Conclusions}
\label{ch:conclusion}

The Myriad X accelerator, when used in Intel's Neural Compute Stick 2 and programmed with the OpenVINO toolkit, is a loosely coupled solution that struggles to deliver high-end performance when utilizing it to do algorithmic transformations for error-tolerant applications. One of the main disadvantages of the Myriad X VPU is that is optimized for visual computing applications which mainly utilize convolutional neural networks. The tested applications, a subset of the AxBench test-suite, were not designed for this type of neural processing unit and as a result the code had to be heavily adapted to make it compatible with the OpenVINO toolkit. CNNs had to be used in an unconventional way to be able to do inference for multiple inputs in parallel in order to overcome the low performance when doing the inference for each item at a time.

To fully utilize the potential of the NCS2 it is required to have a USB3 interface, inference results are 4.6x slower when running in platforms that only support USB2. For even better performance a faster interface is needed. Intel provides PCI-E cards with Myriad X VPUs but testing those will be left for future work. The best performance gains of using the NCS2 to accelerate error-tolerant applications are seen when the device where the code is running has limited computing power capacity, and has a USB3 interface, such as the Raspberry Pi 4. Even there, only two out of five applications saw performance gains compared to the original procedural code.

There are multiple approximate computing optimization techniques that can be implemented to improve the performance and energy efficiency of neural networks such as reducing the input size, reducing the depth of CNNs, reducing the amount of neurons in a fully-connected NN, pruning the networks, using quantization to change the number representation type and number of bits and use separable convolutions in CNNs. It was demonstrated that when applying these techniques when running a neural network using the OpenVINO toolkit in the Myriad X VPU, the performance and the energy efficiency can be improved more than 2x. Also it was demonstrated that performance using an external accelerator such as the NCS2 can be improved up to 10x when compared to the same application running in the CPU of a Rapsberry Pi 4.

\section{Future Work}

In the first part of this work, algorithmic transformations were done to error-tolerant applications and mixed results were obtained. Applications were the code being substituted was simple or small saw degradation but more complex applications saw some improvement when transformed. More applications could be tested using the methods seen in this work to have a greater range of applications and have a broader perspective of what works and what does not work with this technique.

For the second part of the work, three classification applications were tested. Two saw minimal change when applying optimization techniques to improve their performance. Only one saw significant changes and this was because it was the most complex application with the biggest networks and as such, was not bottle-necked by data transfers in the NCS2. More complex applications could be tested using these techniques, this will solidify the results for the scene classification applications. For this work, only classification applications were tested, other type of applications and neural networks could be tested as well. Some examples of different networks could be recurrent neural networks and feedforward neural networks. Other examples of applications can include object detection, image segmentation, text generation, speech synthesis, face detection, etc.
