\chapter*{Abstract}
\thispagestyle{empty}

Machine learning (ML) consistes of algorithms that improve automatically over time by the use of data. This data consists of the inputs and the desired outputs of a specific problem. The machine learning algorithms learn how to fit a neural network to this data in a way that when similar data is fed to it it will give an output which should be correct most of the time.

Approximate computing is a design paradigm with the objective of improving energy efficiency and performance. The trade off is that computation results will have an error associated to them. Usually approximate computing is used in applications where these errors are not a deal breaker and it can be tolerated.

Since machine learning is naturally an error-prone computing paradigm it makes sense to explore how it fits into the approximate computing world. This work focuses in bringing together both machine learning and approximate computing. It explores how off-the-shelf neural accelerators and tools can be used in low-power and embedded environments to leverage the advantages of machine learning in approximate computing applications, and, in the other hand, how can machine learning applications can be accelerated by using approximate computing techniques.

In the first part of this work algorithmic transformations are proposed to take advantage of an off-the-shelf neural accelerator to improve the performance of error-tolerant applications. The results show that two out of the five applications tested can be accelerated maintaining an error under 10\%.

The second part of this work shows how some approximate computing techniques can accelerate machine learning applications and only slightly reduce the accuracy of the neural network. It also shows how using a neural accelerator can greatly improve the energy efficiency of machine learning applications up to 10 times in a low-power environment.

\bigskip

% en español e inglés

\textbf{Keywords:} approximate computing, machine learning, neural accelerator

\cleardoublepage

%%% Local Variables: 
%%% mode: latex
%%% TeX-master: "main"
%%% End: 
